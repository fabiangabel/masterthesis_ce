\documentclass[article,type=msc,colorback,accentcolor=tud2a]{tudthesis}

%%%%%%%%%%%%%%%%%%%%%%%%%%%%%%%%%%%%%%%%%%%%%%%%%%%%%%%%%%%%%%%%%%%%%%%
%\usepackage{ngerman}
\usepackage[english]{babel} 


\usepackage[utf8]{inputenc} %input and fontencoding
\usepackage[T1]{fontenc}


\usepackage{nomencl} %for printing a list of symbols

\usepackage{amsmath,amsthm} %math environments

\usepackage{hyperref} %use hyperlinks in document
\usepackage{url} %display urls as such

%%%%%%%%%%%%%%%%%%%%%%%%%%%%%%%%%%%%%%%%%%%%%%%%%%%%%%%%%%%%%%%%%%%%%%%

\newcommand{\getmydate}{%
  \ifcase\month%
    \or Januar\or Februar\or M\"arz%
    \or April\or Mai\or Juni\or Juli%
    \or August\or September\or Oktober%
    \or November\or Dezember%
  \fi\ \number\year%
}

\begin{document}
\thesistitle{Implementation and Performance Analyses of a Highly Efficient Algorithm for Pressure-Velocity Coupling}{Implementierung und Untersuchung einer hoch effizienten Methode zur Druck-Geschwindigkeits-Kopplung}
  \author{Fabian Nuraddin Alexander Gabel}
  \referee{Prof. Dr. rer. nat. Michael Schäfer}{Dipl.-Ing Ulrich Falk}
  \department{Studienbereich CE}
  \group{FNB}
% \tuprints{12345}{1234}
  \makethesistitle
  \affidavit{F. Gabel}
  \tableofcontents

\printnomenclature
  \section{Introduction}

  This thesis is about. 

  \section{Fundamentals of Continuum Physics for Thermo-Hydrodynamical Problems}

    \subsection{Navier-Stokes Equations for Incompressible Flows}

      \subsubsection{Cauchy-Equations}
      \subsubsection{Newtonian Fluids}
      \subsubsection{Final Form of the Navier-Stokes Equations - Conservative and Nonconservative Form}

    \subsection{Energy Equation}

      \subsubsection{Generic Scalar Transport Equation}

    \subsection{Bouyancy Driven Flows - the Boussinesq Approximatio}
      
      Talk about natural and forced convection. Differences for the solver algorithm. (s.a.) Peric P447
      Talk abourt flows with variation in fluid properties -> mms has to map this behaviour (bouyancy force driven, i.e. naturally convected fluid)
      Also talk about nondimensional values like prandtl number, rayleigh and reynolds

  \section{Finite Volume Methods for Incompressible Flows - Their Theoretical Basics and Their Realisation in Code}

    \subsection{Fundamentals of Discretisation}
      
      \subsubsection{Numerical Grid}
      \subsubsection{Approximation of Integrals}

    \subsection{Discretisation of the Momentum Balance}
      
      \subsubsection{Semi Discretized Linearized Form of the Navier-Stokes Equations}
      \subsubsection{Treatment of Nonorthogonalities}
      \subsubsection{Calculation of Mass Flux - Rhie-Chow Interpolation}
      \subsubsection{Discretization of the Convective Term}
      \subsubsection{Discretization of the Diffusive Term}
      \subsubsection{Discretisation of the Source Term}
      \subsubsection{Assembly of Linear Systems - Final Form of Equations}

    \subsection{Discretisation of the Generic Transport Equation}

    \subsection{Segregated Methods - the SIMPLE-Algorithm}
      
      \subsubsection{Pressure Correction Equation}
      \subsubsection{Characteristical Properties of Projection Methods}

        Underrelaxation, slow convergence, inner iterations outer iterations, relative tolerances
    \subsection{Boundary Conditions on Domain and Block Boundaries}
      
      \subsubsection{Dirichlet Boundary Condition}
      \subsubsection{Neumann Boundary Condition}
      \subsubsection{Symmetry Boundary Condition}
      \subsubsection{Wall Boundary Condition}
      \subsubsection{Block Boundary Condition}
      
    \subsection{Coupled Solution of the Navier-Stokes Equations}

      \subsubsection{Discretization of the Navier-Stokes Equations}
      \subsubsection{Differences to the Segregated Approach - Implicit Coupling of Velocities, Pressure and Temperature}

          Implicit treatment of Pressure Gradient, Implicit Treatment of Temperature possible, boussinesq approximation brings maximal coupling. Temperature dependent densities also possible

      \subsubsection{Assembly of Linear System}

      \subsubsection{Boundary Conditions}

      \subsection{Characteristical Properties of the Fully Coupled Solution Approach}

        Bad condition, singularity, faster convergence, coupling in bouyancy flows (s.a. Peric page 448, Galpin Raithby)

      \subsection{Numerical Solution of Linear Systems}

        \subsubsection{Stone's SIP Solver}

          Basic Idea as in Schäfer or Peric

        \subsubsection{Krylov Subspace Methods}

          General concept of cyclic vector spaces of \(\mathbb{R}^n\), name some representative ksp algorithms, importance of preconditioning, not as detailed as in bachelor thesis

  \section{CAFFA Framework}

    \subsection{PETSc Framework}
        Keep in mind not to copy the manual but
      \subsubsection{About PETSc}

        Bell Prize, MPI Programming

      \subsubsection{Basic Data Types}

        Vec,Mat (Different Matrix Types and Their effect on complex methods)

      \subsubsection{KSP and PC Objects and Their Usage}

        Singularities

      \subsubsection{Profiling}

        Petsc Log 

      \subsubsection{Common Errors}

        Optimization, Interfaces, Compiler Erros not helpful, Preallocation vs. Mallocs

    \subsection{Grid Generation and Conversion}

      Generation of block structured grids, neighbouring relations are represented by a special type of boundary conditions
    \subsection{Preprocessing}
    Matching algorithm - the idea behind clipper and the used projection technique; alt.: Opencascade. Efficient calculation of values for discretization.
    \subsection{CAFFA3D}

      \subsubsection{MPI Programming Model}
        Basic idea of distributed memory programming model, emphasize the differences to shared memory model
      \subsubsection{Indexing of Variables and Treatment of Boundary Values}
      Describe MatZeroValues and how it is used to simplify the code. Compliance of PETSc zero based indexing and CAFFA indexing which considers boundary values. Problems with boundary entries
      \subsubsection{Field Interlacing}
      Realization through special arrangement of variables and the use of index sets (subvector objects) and/or preprocessor directives. Advantages (there was a paper I cited in my thesis)
      \subsubsection{Domain Decomposition, Ghosting and Parallel Matrix Assembly}

        Ghost values are stored in local representations of the global vector (state the mapping for those entries). Matrix coefficients are calculated on one processor and sent to the neighbour. Preallocation as crucial aspect for program performance. Present a simple method for balancing the matrix related load by letting PETSc take care of matrix distribution.

    \subsection{Postprocessing}
    
      Visualization of Results with Paraview and Tecplot

  \section{Verification of CAFFA}
    
    Refer to next section for Validation of CAFFA

    \subsection{Theoretical Discretisation Error}
      present the Taylor-Series Expansion
    \subsection{Method of Manufactured Solutions}
      basically sum up the important points of salari's technical report, symmetry of solution/domain/grid is bad
      point out that mms is not able to detect errors in the physical model

    \subsection{Exact and Manufactured Solutions for the Navier-Stokes Equations and the Energy Equation}
    Not always there is an exact solution. Divergence free approach. Presentation of the used manufactured solution. What if solution is not divergence free? Derivation of equations and modifications to continuity equation. analyze the problem of too complicated manufactured solutions. also use temperature dependent density function
    \begin{itemize}
      \item \url{http://scicomp.stackexchange.com/questions/6943/manufactured-solutions-for-incompressible-navier-stokes-how-to-find-divergenc}
      \item \url{http://link.springer.com/article/10.1007/BF00948290}
      \item \url{http://physics.stackexchange.com/questions/60476/exact-solutions-to-the-navier-stokes-equations}
      \item \url{http://www.annualreviews.org/doi/pdf/10.1146/annurev.fl.23.010191.001111}
    \end{itemize}
    \subsection{Measurement of Error and Calculation of Order}
      Different error measures (L2-Norm,completeness of function space, consistency etc.)

  \section{Comparison of Solver Concepts}
  
    \subsection{Impact on Convergence Behaviour on Blockstructured Grids}

      Show how the implicit treatment of block boundaries maintains (high) convergence rates. Plot Residual over number of iterations.

    \subsection{Parallel Performance}
      \subsubsection{Cluster Hardware and Used Software}
        \begin{itemize}
          \item Mem Section and processes in between islands (calculating across islands)
          \item Versioning information (PETSc,INTEL COMPILERS,CLIPPER,MPI IMPLEMENTATION,BLAS/LAPACK)
          \item Software not designed to perform well on desktop PCs.
        \end{itemize}

      \subsubsection{Measures of Performance}
        \begin{itemize}
          \item Maße definieren
          \item Nochmal Hager,Wellein studieren
          \item Guidelines for measuring performance (bias through system processes or user interaction), only measure calculation time do not consider I/O in the beginning and the end
          \item Cite Schäfer and Peric with their different indicators for parallel efficiency, load balancing and numerical efficiency
        \end{itemize}
      \subsubsection{Preiliminary Upper Bounds on Performance - the STREAM Benchmark}
        Pinning of processes, preiliminary constraints by hardware and operating systems, identification of bottlenecks, history and results of STREAM. Bandwith as Bottleneck. Petsc Implementation of STREAM
      \subsection{Discussion of Results for Parallel Efficiency}
      \subsection{Speedup Measurement for Analytic Testcases}

    \subsection{Application to Testcases with Varying Degree of Non-Linearity}
      
      As Peric says I want to prove that the higher the nonlinearity of NS, the better relative convergence rates can be achieved with a coupled solver. Fi

      \subsubsection{Transport of a Passive Scalar - Forced Convection}
      \subsubsection{Bouyancy Driven Flow - Natural Convection}
      \subsubsection{Flow with Temperature Dependent Density - A Highly Nonlinear Testcase}

    \subsection{Comparison of Solver Concepts in Realistic Scenarios}
      Also consider simple load balancing by distributing matrix rows equally
      
      \subsubsection{Flow around a cylinder 3d - stationary}
      \subsubsection{Flow around a cylinder 3d - instationary}
        \begin{itemize}
          \item\url{http://www.featflow.de/en/benchmarks/cfdbenchmarking/flow/dfg_flow3d/dfg_flow3d_configuration.html}
        \end{itemize}

      \subsubsection{Heat-Driven Cavity Flow}
        \begin{itemize}
          \item \url{http://www.featflow.de/en/benchmarks/cfdbenchmarking/mit_benchmark.html}
        \end{itemize}
      
  \section{Conclusion and Outlook}
    Turbulence, Multiphase, GPU-Accelerators, Load-Balancing, dynamic mesh refinement, Counjugate Heat Transfer with other requirements for the numerical grid, grid movement

\end{document}
