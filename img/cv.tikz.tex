    \begin{tikzpicture}
	%%% Edit the following coordinate to change the shape of your
	%%% cuboid
      
	%% Vanishing points for perspective handling
	\coordinate (P1) at (-25cm,6.5cm); % left vanishing point (To pick)
	\coordinate (P2) at (18cm,5.5cm); % right vanishing point (To pick)

	%% (A1) and (A2) defines the 2 central points of the cuboid
	\coordinate (A1) at (0em,0cm); % central top point (To pick)
	\coordinate (A2) at (0em,-2.5cm); % central bottom point (To pick)

	%% (A3) to (A8) are computed given a unique parameter (or 2) .8
	% You can vary .8 from 0 to 1 to change perspective on left side
	\coordinate (A3) at ($(P1)!.9!(A2)$); % To pick for perspective 
	\coordinate (A4) at ($(P1)!.9!(A1)$);

	% You can vary .8 from 0 to 1 to change perspective on right side
	\coordinate (A7) at ($(P2)!.9!(A2)$);
	\coordinate (A8) at ($(P2)!.9!(A1)$);

	%% Automatically compute the last 2 points with intersections
	\coordinate (A5) at
	  (intersection cs: first line={(A8) -- (P1)},
			    second line={(A4) -- (P2)});
	\coordinate (A6) at
	  (intersection cs: first line={(A7) -- (P1)}, 
			    second line={(A3) -- (P2)});

	%%% Depending of what you want to display, you can comment/edit
	%%% the following lines

	%% Possibly draw back faces

	\fill[gray!90] (A2) -- (A3) -- (A6) -- (A7) -- cycle; % face 6
	\node at (barycentric cs:A2=1,A3=1,A6=1,A7=1) { $S_b$};
	
	\fill[gray!50] (A3) -- (A4) -- (A5) -- (A6) -- cycle; % face 3
	\node at (barycentric cs:A3=1,A4=1,A5=1,A6=1) { $S_w$};
	
	\fill[gray!30] (A5) -- (A6) -- (A7) -- (A8) -- cycle; % face 4
	\node at (barycentric cs:A5=1,A6=1,A7=1,A8=1) { $S_n$};
	
	\draw[thick,dashed] (A5) -- (A6);
	\draw[thick,dashed] (A3) -- (A6);
	\draw[thick,dashed] (A7) -- (A6);

	%% Possibly draw front faces

	% \fill[orange] (A1) -- (A8) -- (A7) -- (A2) -- cycle; % face 1
	  \node at (barycentric cs:A1=1,A8=1,A7=1,A2=1) { $S_e$};
	\fill[gray!50,opacity=0.2] (A1) -- (A2) -- (A3) -- (A4) -- cycle; % f2
	\node at (barycentric cs:A1=1,A2=1,A3=1,A4=1) { $S_s$};
	\fill[gray!90,opacity=0.2] (A1) -- (A4) -- (A5) -- (A8) -- cycle; % f5
	\node at (barycentric cs:A1=1,A4=1,A5=1,A8=1) { $S_t$};

	%% Possibly draw front lines
	\draw[thick] (A1) -- (A2);
	\draw[thick] (A3) -- (A4);
	\draw[thick] (A7) -- (A8);
	\draw[thick] (A1) -- (A4);
	\draw[thick] (A1) -- (A8);
	\draw[thick] (A2) -- (A3);
	\draw[thick] (A2) -- (A7);
	\draw[thick] (A4) -- (A5);
	\draw[thick] (A8) -- (A5);
	
	% Possibly draw points
	% (it can help you understand the cuboid structure)
	%\foreach \i in {1,2,...,8}
	%{
	%  \draw[fill=black] (A\i) circle (0.15em)
	%    node[above right] {\tiny \i};
	%}
	%\draw[fill=black] (P1) circle (0.1em) node[below] {\tiny p1};
	%\draw[fill=black] (P2) circle (0.1em) node[below] {\tiny p2};
\end{tikzpicture}
