  \section{Finite Volume Method for Incompressible Flows -- Theoretical Basics}

  This section deals with the fundamentals of the numerical solution via a finite volume method of the formerly presented set of partial differential equations. The focus of this section is, to provide an overview over the methods to be used in the present thesis. The information contained in this section is based on (Peric,Schäfer,Muzaferja,Jsak). The overview starts by mentioning the different grid types to be used and the discretization techniques to be applied. On the basis of integral formulations of the equations to be solved, the therein contained integrals and differential operators have to be discretized. Since the accuracy of the default concepts for discretizing differential operators degrades with decreasing grid quality, this chapter furthermore presents different approaches to handle corrections for cases in which the cause of degrading grid quality is increased non-orthogonality. 
  
 The goal of the finite volume method is to provide algebraic equations which can be used to determine an approximate solution of a partial differential equation. This system of linear algebraic equations can be solved by means of algorithms to be presented in the end of this section. However since the Navier-Stokes equations are in general non-linear an intermediate step has to be taken, by linearizing the discrete equations. This leads to the need of an iteration process, the \textit{Picard iteration}, which will be explained briefly.
      
  \subsection{Numerical Grid}

    In this subsection a brief overview of the general grid structure to be used in the present thesis is given. The main idea behind finite volume methods is to solve partial differential equations by integrating them over the specified continuous problem domain and dividing this domain into a finite number of subdomains, the so called control volumes. The result of the this finite partition of a continuous problem domain is called the numerical grid. The grid consists of a finite number of grid cells which represent the boundaries of a discrete domain of integration. 

    Regarding the treatment of domain boundaries, and the ordering of the cells within the problem domain different types of numerical grids can be distinguished. The present thesis makes use of so called block structured grids with hexahedron cells. A structured grid is characterized by a constant amount of of grid cells in each coordinate direction. The high regularity of structured grids benefits the computational efficiency of algorithms to be used on this type of grid. A block structured grid consists of different grid blocks, of which each considered individually is structured but considered globally the grid is unstructured. An example of a block structured grid with distinguishable grid blocks is given in FIGURE. The use of block structured grids is motivated by the need to increase the adaptivity of structured grids by maintaining high computational efficiency. 

    Inside a structured grid block, cells with the shape of hexahedrons are used. In addition to the geometric boundaries of each control volume a numerical grid also provides a mapping that assigns to each control volume with index \(P\) a set of indexes of neighbouring control volumes \(NB(P):=\{W,S,B,T,N,E\}\), which are named after the geographic directions. The faces of each hexahedral control volume represent the mentioned geometric boundaries. 

    \begin{itemize}
      \item talk about grid quality
      \item talk about local refinement
    \end{itemize}
    
    \begin{tikzpicture}
	%%% Edit the following coordinate to change the shape of your
	%%% cuboid
      
	%% Vanishing points for perspective handling
	\coordinate (P1) at (-7cm,1.5cm); % left vanishing point (To pick)
	\coordinate (P2) at (8cm,1.5cm); % right vanishing point (To pick)

	%% (A1) and (A2) defines the 2 central points of the cuboid
	\coordinate (A1) at (0em,0cm); % central top point (To pick)
	\coordinate (A2) at (0em,-4cm); % central bottom point (To pick)

	%% (A3) to (A8) are computed given a unique parameter (or 2) .8
	% You can vary .8 from 0 to 1 to change perspective on left side
	\coordinate (A3) at ($(P1)!.7!(A2)$); % To pick for perspective 
	\coordinate (A4) at ($(P1)!.7!(A1)$);

	% You can vary .8 from 0 to 1 to change perspective on right side
	\coordinate (A7) at ($(P2)!.65!(A2)$);
	\coordinate (A8) at ($(P2)!.65!(A1)$);

	%% Automatically compute the last 2 points with intersections
	\coordinate (A5) at
	  (intersection cs: first line={(A8) -- (P1)},
			    second line={(A4) -- (P2)});
	\coordinate (A6) at
	  (intersection cs: first line={(A7) -- (P1)}, 
			    second line={(A3) -- (P2)});

	%%% Depending of what you want to display, you can comment/edit
	%%% the following lines

	%% Possibly draw back faces

	\fill[gray!90] (A2) -- (A3) -- (A6) -- (A7) -- cycle; % face 6
	\node at (barycentric cs:A2=1,A3=1,A6=1,A7=1) { $S_b$};
	
	\fill[gray!50] (A3) -- (A4) -- (A5) -- (A6) -- cycle; % face 3
	\node at (barycentric cs:A3=1,A4=1,A5=1,A6=1) { $S_w$};
	
	\fill[gray!30] (A5) -- (A6) -- (A7) -- (A8) -- cycle; % face 4
	\node at (barycentric cs:A5=1,A6=1,A7=1,A8=1) { $S_n$};
	
	\draw[thick,dashed] (A5) -- (A6);
	\draw[thick,dashed] (A3) -- (A6);
	\draw[thick,dashed] (A7) -- (A6);

	%% Possibly draw front faces

	% \fill[orange] (A1) -- (A8) -- (A7) -- (A2) -- cycle; % face 1
	  \node at (barycentric cs:A1=1,A8=1,A7=1,A2=1) { $S_e$};
	\fill[gray!50,opacity=0.2] (A1) -- (A2) -- (A3) -- (A4) -- cycle; % f2
	\node at (barycentric cs:A1=1,A2=1,A3=1,A4=1) { $S_s$};
	\fill[gray!90,opacity=0.2] (A1) -- (A4) -- (A5) -- (A8) -- cycle; % f5
	\node at (barycentric cs:A1=1,A4=1,A5=1,A8=1) { $S_t$};

	%% Possibly draw front lines
	\draw[thick] (A1) -- (A2);
	\draw[thick] (A3) -- (A4);
	\draw[thick] (A7) -- (A8);
	\draw[thick] (A1) -- (A4);
	\draw[thick] (A1) -- (A8);
	\draw[thick] (A2) -- (A3);
	\draw[thick] (A2) -- (A7);
	\draw[thick] (A4) -- (A5);
	\draw[thick] (A8) -- (A5);
	
	% Possibly draw points
	% (it can help you understand the cuboid structure)
	%\foreach \i in {1,2,...,8}
	%{
	%  \draw[fill=black] (A\i) circle (0.15em)
	%    node[above right] {\tiny \i};
	%}
	%\draw[fill=black] (P1) circle (0.1em) node[below] {\tiny p1};
	%\draw[fill=black] (P2) circle (0.1em) node[below] {\tiny p2};
\end{tikzpicture}

\newpage
\begin{tikzpicture}
	\coordinate (P1) at (-20cm,6.5cm); % left vanishing point (To pick)
	\coordinate (P2) at (18cm,5.5cm); % right vanishing point (To pick)

	\coordinate (A1) at (0em,0cm); % central top point (To pick)
	\coordinate (A2) at (0em,1.5cm); % central bottom point (To pick)
	\coordinate (A3) at (0em,3cm); % central bottom point (To pick)

	\coordinate (A4) at ($(P1)!.925!(A1)$); 
	\coordinate (A5) at ($(P1)!.925!(A2)$);
	\coordinate (A6) at ($(P1)!.925!(A3)$);

	\coordinate (A7) at ($(P1)!.85!(A1)$); 
	\coordinate (A8) at ($(P1)!.85!(A2)$);
	\coordinate (A9) at ($(P1)!.85!(A3)$);

	\coordinate (A10) at ($(P2)!.925!(A1)$); 
	\coordinate (A11) at ($(P2)!.925!(A2)$);
	\coordinate (A12) at ($(P2)!.925!(A3)$);

	\coordinate (A19) at ($(P2)!.85!(A1)$); 
	\coordinate (A20) at ($(P2)!.85!(A2)$);
	\coordinate (A21) at ($(P2)!.85!(A3)$);

	\coordinate (A13) at
	  (intersection cs: first line={(A10) -- (P1)},
			    second line={(A4) -- (P2)});
	\coordinate (A14) at
	  (intersection cs: first line={(A11) -- (P1)}, 
			    second line={(A5) -- (P2)});
	\coordinate (A15) at
	  (intersection cs: first line={(A12) -- (P1)}, 
			    second line={(A6) -- (P2)});

	\coordinate (A16) at
	  (intersection cs: first line={(A10) -- (P1)},
			    second line={(A7) -- (P2)});
	\coordinate (A17) at
	  (intersection cs: first line={(A11) -- (P1)}, 
			    second line={(A8) -- (P2)});
	\coordinate (A18) at
	  (intersection cs: first line={(A12) -- (P1)}, 
			    second line={(A9) -- (P2)});

	\coordinate (A22) at
	  (intersection cs: first line={(A19) -- (P1)},
			    second line={(A4) -- (P2)});
	\coordinate (A23) at
	  (intersection cs: first line={(A20) -- (P1)}, 
			    second line={(A5) -- (P2)});
	\coordinate (A24) at
	  (intersection cs: first line={(A21) -- (P1)}, 
			    second line={(A6) -- (P2)});

	\coordinate (A25) at
	  (intersection cs: first line={(A19) -- (P1)},
			    second line={(A7) -- (P2)});
	\coordinate (A26) at
	  (intersection cs: first line={(A20) -- (P1)}, 
			    second line={(A8) -- (P2)});
	\coordinate (A27) at
	  (intersection cs: first line={(A21) -- (P1)}, 
			    second line={(A9) -- (P2)});

        \foreach \i in {1,2,3,4,5,6,7,8,9,
                        10,11,12,15,18,
                        19,20,21,24,27}
        {
	 %\draw[fill=black] (A\i) circle (0.15em) ;
        }

        %Vertical Lines
        \draw (A1) -- (A2) -- (A3);
        \draw (A4) -- (A5) -- (A6) -- (A15) -- (A24);
        \draw (A7) -- (A8) -- (A9) -- (A18) -- (A27);
        \draw (A12) -- (A15) -- (A18);
        \draw (A21) -- (A24) -- (A27);

        %Horizontal Lines
        \draw (A7) -- (A4) -- (A1);
        \draw (A8) -- (A5) -- (A2);
        \draw (A9) -- (A6) -- (A3) -- (A12) -- (A21);

	\coordinate (B1) at (0em,0cm);
	\coordinate (B2) at (0em,1cm);
	\coordinate (B3) at (0em,2cm);
	\coordinate (B4) at (0em,3cm);

	\coordinate (B5) at ($(P1)!1.1!(B1)$); 
	\coordinate (B6) at ($(P1)!1.1!(B2)$);
	\coordinate (B7) at ($(P1)!1.1!(B3)$);
	\coordinate (B8) at ($(P1)!1.1!(B4)$);

	\coordinate (B9)  at ($(P1)!1.2!(B1)$); 
	\coordinate (B10) at ($(P1)!1.2!(B2)$);
	\coordinate (B11) at ($(P1)!1.2!(B3)$);
	\coordinate (B12) at ($(P1)!1.2!(B4)$);

	\coordinate (B13)  at ($(P1)!1.3!(B1)$); 
	\coordinate (B14) at ($(P1)!1.3!(B2)$);
	\coordinate (B15) at ($(P1)!1.3!(B3)$);
	\coordinate (B16) at ($(P1)!1.3!(B4)$);

	\coordinate (B17) at ($(P2)!0.95!(B1)$); 
	\coordinate (B18) at ($(P2)!0.95!(B2)$);
	\coordinate (B19) at ($(P2)!0.95!(B3)$);
	\coordinate (B20) at ($(P2)!0.95!(B4)$);

        \coordinate (B33) at ($(P2)!.90!(B1)$); 
        \coordinate (B34) at ($(P2)!.90!(B2)$);
        \coordinate (B35) at ($(P2)!.90!(B3)$);
        \coordinate (B36) at ($(P2)!.90!(B4)$);

        \coordinate (B49) at ($(P2)!.85!(B1)$); 
        \coordinate (B50) at ($(P2)!.85!(B2)$);
        \coordinate (B51) at ($(P2)!.85!(B3)$);
        \coordinate (B52) at ($(P2)!.85!(B4)$);

	\coordinate (B21) at
	  (intersection cs: first line={(B17) -- (P1)},
			    second line={(B5) -- (P2)});
	\coordinate (B22) at
	  (intersection cs: first line={(B18) -- (P1)},
			    second line={(B6) -- (P2)});
	\coordinate (B23) at
	  (intersection cs: first line={(B19) -- (P1)},
			    second line={(B7) -- (P2)});
	\coordinate (B24) at
	  (intersection cs: first line={(B20) -- (P1)},
			    second line={(B8) -- (P2)});

	\coordinate (B25) at
	  (intersection cs: first line={(B17) -- (P1)},
			    second line={(B9) -- (P2)});
	\coordinate (B26) at
	  (intersection cs: first line={(B18) -- (P1)},
			    second line={(B10) -- (P2)});
	\coordinate (B27) at
	  (intersection cs: first line={(B19) -- (P1)},
			    second line={(B11) -- (P2)});
	\coordinate (B28) at
	  (intersection cs: first line={(B20) -- (P1)},
			    second line={(B12) -- (P2)});

	\coordinate (B29) at
	  (intersection cs: first line={(B17) -- (P1)},
			    second line={(B13) -- (P2)});
	\coordinate (B30) at
	  (intersection cs: first line={(B18) -- (P1)},
			    second line={(B14) -- (P2)});
	\coordinate (B31) at
	  (intersection cs: first line={(B19) -- (P1)},
			    second line={(B15) -- (P2)});
	\coordinate (B32) at
	  (intersection cs: first line={(B20) -- (P1)},
			    second line={(B16) -- (P2)});

	\coordinate (B37) at
	  (intersection cs: first line={(B33) -- (P1)},
			    second line={(B5) -- (P2)});
	\coordinate (B38) at
	  (intersection cs: first line={(B34) -- (P1)},
			    second line={(B6) -- (P2)});
	\coordinate (B39) at
	  (intersection cs: first line={(B35) -- (P1)},
			    second line={(B7) -- (P2)});
	\coordinate (B40) at
	  (intersection cs: first line={(B36) -- (P1)},
			    second line={(B8) -- (P2)});

	\coordinate (B41) at
	  (intersection cs: first line={(B33) -- (P1)},
			    second line={(B9) -- (P2)});
	\coordinate (B42) at
	  (intersection cs: first line={(B34) -- (P1)},
			    second line={(B10) -- (P2)});
	\coordinate (B43) at
	  (intersection cs: first line={(B35) -- (P1)},
			    second line={(B11) -- (P2)});
	\coordinate (B44) at
	  (intersection cs: first line={(B36) -- (P1)},
			    second line={(B12) -- (P2)});

	\coordinate (B45) at
	  (intersection cs: first line={(B33) -- (P1)},
			    second line={(B13) -- (P2)});
	\coordinate (B46) at
	  (intersection cs: first line={(B34) -- (P1)},
			    second line={(B14) -- (P2)});
	\coordinate (B47) at
	  (intersection cs: first line={(B35) -- (P1)},
			    second line={(B15) -- (P2)});
	\coordinate (B48) at
	  (intersection cs: first line={(B36) -- (P1)},
			    second line={(B16) -- (P2)});

	\coordinate (B53) at
	  (intersection cs: first line={(B49) -- (P1)},
			    second line={(B5) -- (P2)});
	\coordinate (B54) at
	  (intersection cs: first line={(B50) -- (P1)},
			    second line={(B6) -- (P2)});
	\coordinate (B55) at
	  (intersection cs: first line={(B51) -- (P1)},
			    second line={(B7) -- (P2)});
	\coordinate (B56) at
	  (intersection cs: first line={(B52) -- (P1)},
			    second line={(B8) -- (P2)});

	\coordinate (B57) at
	  (intersection cs: first line={(B49) -- (P1)},
			    second line={(B9) -- (P2)});
	\coordinate (B58) at
	  (intersection cs: first line={(B50) -- (P1)},
			    second line={(B10) -- (P2)});
	\coordinate (B59) at
	  (intersection cs: first line={(B51) -- (P1)},
			    second line={(B11) -- (P2)});
	\coordinate (B60) at
	  (intersection cs: first line={(B52) -- (P1)},
			    second line={(B12) -- (P2)});

	\coordinate (B61) at
	  (intersection cs: first line={(B49) -- (P1)},
			    second line={(B13) -- (P2)});
	\coordinate (B62) at
	  (intersection cs: first line={(B50) -- (P1)},
			    second line={(B14) -- (P2)});
	\coordinate (B63) at
	  (intersection cs: first line={(B51) -- (P1)},
			    second line={(B15) -- (P2)});
	\coordinate (B64) at
	  (intersection cs: first line={(B52) -- (P1)},
			    second line={(B16) -- (P2)});


        %Front lines
        \draw (B13) -- (B29) -- (B45) -- (B61);
        \draw (B1) -- (B5) -- (B9) -- (B13);
        \draw (B2) -- (B6) -- (B10) -- (B14);
        \draw (B3) -- (B7);
        \draw (B14) -- (B62);

        %Vertical lines
        \draw (B5) -- (B6) -- (B7) -- (B8);
        \draw (B29) -- (B30);
        \draw (B45) -- (B46) -- (B47) -- (B48);
        \draw (B61) -- (B62) -- (B63) -- (B64);
        \draw (B13) -- (B14);
        \draw (B9) -- (B10);
        \draw (B38) -- (B40);
        \draw (B22) -- (B24);
        \draw (B42) -- (B44);

        %horizontal lines
        \draw (B22) -- (B26) -- (B30);

        %linien in die tiefe
        \draw (B6) -- (B22); 
        \draw (B7) -- (B39);
        \draw (B10) -- (B42);
        \draw (B47) -- (B63);
        \draw (B48) -- (B64);
        \draw (B44) -- (B60);

        \draw (B8) -- (B24) -- (B40) -- (B56);

        %Top lines (to east)
        \draw (B4) -- (B8);
        \draw (B20) -- (B24);
        \draw (B36) -- (B40) -- (B44) -- (B48);
        \draw (B52) -- (B56) -- (B60) -- (B64);
        \draw (B38) -- (B46);
        \draw (B39) -- (B47);

        %fill out one CV
        \fill[gray!50,opacity=0.7] (B26) -- (B27) -- (B23) -- (B22) -- cycle;
        \fill[gray!80,opacity=0.7] (B42) -- (B43) -- (B39) -- (B38) -- cycle;
        \fill[gray!80,opacity=1.0] (B22) -- (B38) -- (B39) -- (B23) -- cycle; 
        \fill[gray!50,opacity=0.7] (B26) -- (B42) -- (B43) -- (B27) -- cycle; 
        \fill[gray!50,opacity=1.0] (B22) -- (B26) -- (B42) -- (B38) -- cycle;
        \fill[gray!80,opacity=0.7] (B23) -- (B27) -- (B43) -- (B39) -- cycle;

        \draw[gray!90,dashed] (B8) -- (B16) -- (B48);
        \draw[gray!90,dashed] (B16) -- (B14);
        \draw[gray!90,dashed] (B32) -- (B30);
        \draw[gray!90,dashed] (B12) -- (B10);
        \draw[gray!90,dashed] (B32) -- (B30);
        \draw[gray!90,dashed] (B7) -- (B15) -- (B47);
        \draw[gray!90,dashed] (B23) -- (B31);
        \draw[gray!90,dashed] (B12) -- (B44);
        \draw[gray!90,dashed] (B26) -- (B28);
        \draw[gray!90,dashed] (B24) -- (B32);


        %right lines (vertical)




        \foreach \i in {1,2,3,4,5,6,7,8,9,10,11,12,13,14,15,16,17,18,19,20,21,22,23,24,25,26,27,28,29,30,31,32,33,34,35,36,37,38,39,40,41,42,43,44,45,46,47,48,49,50,51,52,53,54,55,56,57,58,59,60,61,62,63,64}
        {
	 %\draw[fill=black] (B\i) circle (0.15em) ;
        }
        

\end{tikzpicture}


    \subsection{Approximation of Integrals and Derivatives}

    Briefly explain the idea behing the quadrature via the midpoint rule. Talk about central differences and the approximation via the gauss theorem. Maybe talk about the resulting order of the truncation error. 

    \subsection{Treatment of Non-Orthogonality of Grid Cells -- Deferred Correction Approach}

    \begin{itemize}
      \item note that each method should vanish on orthogonal grids
      \item Stick to Jsak and refer to Muzaferja's PHD thesis and Ferziger/Peric. 
      \item Note the discrepancy in Perics book's referral to Muzaferja's thesis. 
      \item Talk about the benefits of each approximation, maybe with example calculation 
      \item Talk about the overall convergence rates
      \item note that this only accelerates the solver. The fully converged solution should be independent of the treatment of the non-orthogonalities. (Proof?)
    \end{itemize}

      \subsubsection{Minimum Correction Approach}
      \subsubsection{Orthogonal Correction Approach}
      \subsubsection{Over-Relaxed Approach}

    \subsection{Numerical Solution of Non-Linear Systems}

      Introduce the concept of the Picard-Iteration as a linearization technique. Introduce the notions of inner and outer iterations. Refer to later chapters when it comes to deferred correction.

    \subsection{Numerical Solution of Linear Systems}

        The discretization techniques to be presented in the following chapter in composition with a Picard iteration transform the approximate solution of a system of partial differential equations into the solution of algebraic equations via linear systems.

       \subsubsection{Stone's SIP Solver}

         Basic Idea as in Schäfer or Peric. Emphasize that this is a very problem specific approach, that cannot be generalized that easily in opposition to the general purpose linear solvers from PETSc. Present ether BiCGStab or GMRES, the one which performs better and is used throughout the thesis.

       \subsubsection{Krylov Subspace Methods}
        \begin{itemize}
          \item General concept of cyclic vector spaces of \(\mathbb{R}^n\), 
          \item talk about bases of krylov subspaces and the arnoldi algorithm, talk about polynomials and linear combinations
          \item mention the two major branches (minimum residual approach, petrov and ritz-galerkin approach) 
          \item name some representative ksp algorithms, importance of preconditioning, not as detailed as in bachelor thesis
          \item in cases there is a nonempty Nullspace what happens?
        \end{itemize}

